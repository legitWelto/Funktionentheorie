\subsection{Komplexe Ableitung und Cauchy-Riemannsche Differentialgleichung}
\textbf{Erinnerung:}\\
Eine Funktion $f:D \to \C, D \subset \C$ ist stetig, wenn sie stetig ist aufgefasst als Abbildung einer Teilmenge $D\subset \R^2$ in den $\R^2$.\\
Eine Teilmenge $D\subset \R^2$ ist offen, falls $\forall q \in D \quad \exists \veps > 0: \cup _\veps(a) = \{z \in \R^2| |z-a| < \veps\} \subset D$.\\
$a \in \C$ ist Häufungspunkt von D, falls $ \forall \veps > 0 \quad \exists z \in D$ mit $|z-a| < \veps$ .\\
Sei $f: D \to \C, L \in \C,$ dann bedeutet die Schreibweise $\begin{displaystyle}
\lim_{z \to a} f(z)=l 
\end{displaystyle}$, dass a ein Häufungspunkt von D ist und $\tilde{f} : D \cup \{a\} \to \C, z\mapsto \left\{
\begin{array}{ll}
	f(z)&z\neq a\\
	l & z=a 
\end{array}\right. $ ist in a stetig


\begin{definition}
Eine Funktion $f: D \to \C, \quad D \subset \C$ offen, heißt \textbf{komplex differenzierbar im Punkt $a\in D$}, falls:
\[
\lim_{z \to a} \frac{f(z) -f(a)}{z-a}
\]
existiert. Falls der Grenzwert existiert schreiben wir für den Grenzwert:
\[
f\prime(a) = \frac{df}{dz}(a)
\]
Wenn f in jedem Punkt komplex differenzierbar ist, dann ist $f\prime$ wieder eine Funktion auf D.
\end{definition}

\begin{bemerkung}
Alternative Formulierung der komplexen Differenzierbarkeit:
Die folgenden Aussagen sind äquivalent:
\begin{itemize}
	\item[1)] 
	f ist in $a\in D$ komplex differenzierbar und hat den Grenzwert l
	\item[2)]
	$\exists f_1$, stetig in a, mit $f(z) = f_1(z)(z-a), f(a) = l$
	\item[3)]
	$\exists g: D \to \C$, stetig in a, mit $f(z) = f(a) + (z-a)l+ g(z), g(a)=0$
	\item[4)]
	$\exists r: D\to \C$, stetig in a, mit $f(z) = f(a) +(z-a)l + r(z), \lim_{z \to a}\frac{r(z)}{z-a}= 0$\\
	Insbesondere folgt: f ist komplex differenzierbar in a $\Rightarrow$ f stetig in a.
\end{itemize}
\end{bemerkung}

\begin{satz}
Seien $f,g: D \to \C, D\subset \C$, offen, komplex differenzierbar auf D, Dann sind, f+g, $\lambda$f ($\lambda \in \C$), $f \cdot g, \frac{f}{g}$ ($g(a) \neq 0$) ebenfalls komplex differenzierbar. Es gelten die üblichen Ableitungsregeln(Ketten-,Produkt-,Quotientenregel).
\end{satz}