\subsection{Komplexe Ableitung und Cauchy-Riemannsche Differentialgleichung}
\textbf{Erinnerung:}\\
Eine Funktion $f:D \to \C, D \subset \C$ ist stetig, wenn sie stetig ist aufgefasst als Abbildung einer Teilmenge $D\subset \R^2$ in den $\R^2$.\\
Eine Teilmenge $D\subset \R^2$ ist offen, falls $\forall q \in D \quad \exists \veps > 0: \cup _\veps(a) = \{z \in \R^2| |z-a| < \veps\} \subset D$.\\
$a \in \C$ ist Häufungspunkt von D, falls $ \forall \veps > 0 \quad \exists z \in D$ mit $|z-a| < \veps$ .\\
Sei $f: D \to \C, L \in \C,$ dann bedeutet die Schreibweise $\begin{displaystyle}
\lim_{z \to a} f(z)=l 
\end{displaystyle}$, dass a ein Häufungspunkt von D ist und $\tilde{f} : D \cup \{a\} \to \C, z\mapsto \left\{
\begin{array}{ll}
	f(z)&z\neq a\\
	l & z=a 
\end{array}\right. $ ist in a stetig


\begin{definition}
Eine Funktion $f: D \to \C, \quad D \subset \C$ offen, heißt \textbf{komplex differenzierbar im Punkt $a\in D$}, falls:
\[
\lim_{z \to a} \frac{f(z) -f(a)}{z-a}
\]
existiert. Falls der Grenzwert existiert schreiben wir für den Grenzwert:
\[
f\prime(a) = \frac{df}{dz}(a)
\]
Wenn f in jedem Punkt komplex differenzierbar ist, dann ist $f\prime$ wieder eine Funktion auf D.
\end{definition}

\begin{bemerkung}
Alternative Formulierung der komplexen Differenzierbarkeit:
Die folgenden Aussagen sind äquivalent:
\begin{itemize}
	\item[1)] 
	f ist in $a\in D$ komplex differenzierbar und hat den Grenzwert l
	\item[2)]
	$\exists f_1$, stetig in a, mit $f(z) = f_1(z)(z-a), f(a) = l$
	\item[3)]
	$\exists g: D \to \C$, stetig in a, mit $f(z) = f(a) + (z-a)l+ g(z), g(a)=0$
	\item[4)]
	$\exists r: D\to \C$, stetig in a, mit $f(z) = f(a) +(z-a)l + r(z), \lim_{z \to a}\frac{r(z)}{z-a}= 0$\\
	Insbesondere folgt: f ist komplex differenzierbar in a $\Rightarrow$ f stetig in a.
\end{itemize}
\end{bemerkung}

\begin{satz}
Seien $f,g: D \to \C, D\subset \C$, offen, komplex differenzierbar auf D, Dann sind, f+g, $\lambda$f ($\lambda \in \C$), $f \cdot g, \frac{f}{g}$ ($g(a) \neq 0$) ebenfalls komplex differenzierbar. Es gelten die üblichen Ableitungsregeln(Ketten-,Produkt-,Quotientenregel).
\end{satz}
\begin{beispiel}
	\leavevmode
	\begin{itemize}
		\item[1)]
		$f: D \to \C, z\mapsto z^n, n \in \Z  $\\
		ist komplex differenzierbar mit $f\prime(z) = n z^{n-1}$ 
		$D = \left\{
		\begin{array}{ll}
		\C &n > 0\\
		\C^{\times} &n \leq 0
		\end{array}\right. $ \\
		Jedes Polynom $ P(z) = \sum_{k=0}^{n} a_k z^k ,a_k \in \C , a_n \ne 0$ ist komplex differenzierbar und es gilt: $P\prime(z) = \sum^{n - 1}_{k=0} k a_k z^{k-1}$ \\
		Sei $Q(z)$ ein weiteres Polynom\\
		$N(Q) = \{z \in \C| Q(z)=0\}$\\
		Dann ist $f :\C \setminus n(Q) \to \C$ komplex differenzierbar\\
		$z \mapsto \frac{P(z)}{Q(z)}$
		\item[2)]
		Sei $(a_n)_{n \in \N}, a_n \in \N$ eine Folge.\\
		Annahme: Die Potenzreihe $f(z) = \sum^\infty_{n=0} a_n(z-a)^n$
		konvergieren in \\
		$B_{R}(a)=\{z \in \C| |z-a|<R\} R>0\}$ \\
		Wir werden später sehen, f ist in jedem Punkt von $B_R(a)$ 
		komplex differenzierbar\\
	\end{itemize}
\end{beispiel}


\begin{bemerkung}\label{bemer}
	\leavevmode
	\begin{itemize}
		\item
		Erinnerung aus Analysis II :\\
		Eine Abbildung von D nach $\R^a, D \subset \R^p$, offen, heißt total
		differenzierbar in $a \in D$, falls es eine $\R$-lineare Abbildung 
		$T: \R^p \to \R^q$ gibt, mit:\\
		$f(x)-f(a) = T(x-a) +r(x), \lim_{x \to a} \frac{r(x)}{|x-a|}= 0$\\
		T ist eindeutig bestimmt und heißt Jacobi-Abbildung von f in a 
		oder Tangentialabbildung von f in a oder totales Differential. Wir 
		schreiben: \\
		$Tf(a) := T$ \\
		$ f: D \rightarrow \C , D \subset \C $ offen ist komplex differenzierbar in $a \in D \Leftrightarrow$ f ist in a total differenzierbar ($\C \tilde{=} \R^2$ ) und die Tangentialabbildung ist von der Form : $Tf(a)z = lz$ mit $l = f\prime(a) \in \C$ 
		\item
		Erinnerung aus der Linearen Algebra:\\
		Für eine $\R$-lineare Abb. $T: \C \to \C$ sind folgende Aussagen äquivalent:
		\begin{itemize}
			\item[1)]
			$\exists l \in \C \text{ mit } Tz = lz$
			\item[2)]
			T ist $\C$-linear
			\item[3)]
			$T(i) = iT(1)$
			\item[4)]
			Die Matrix von T bezüglich der Basis $1(=(1,0))$ und die Form \\
			i(=(0,1)) hat die Form:
			\[
			\begin{bmatrix}
			\a & -\b\\
			\b & \a
			\end{bmatrix}
			\]
		\end{itemize}
		\begin{bew}
			$1 \Leftrightarrow 4$ Wir können die Matrix von t schreiben als:
			\[ \begin{bmatrix}
			a & b \\
			c & d
			\end{bmatrix}
			\rightarrow \begin{bmatrix}
			u \\
			v
			\end{bmatrix} = 
			\begin{bmatrix}
			a & b \\
			c & d
			\end{bmatrix}
			\begin{bmatrix}
			x \\
			y
			\end{bmatrix}\]
			\begin{eqnarray*}
				\text{aus } Tz = lz, l= \a + i\b, z = x + iy = (x,y) \\
				\Rightarrow T(x,y) =& (\a x - \b y, \a y + \b x)\\
				=& \begin{bmatrix}
					a & b \\
					c & d
				\end{bmatrix}
				\begin{bmatrix}
					x \\
					y
				\end{bmatrix}
			\end{eqnarray*}
			
		\end{bew}
		\item
		Erinnerung aus Analysis II:\\
		Die Matrix der Tangentialabbildung ist die Jacobi-Matrix der 
		partiellen Ableitungen.\\
		$f:D \to \C, D\subset \C$ offen, total differenzierbar in $a \in D$.\\
		$f(x,y) = u(x,y) +iv(x,y), z = x+iy=(x,y)$\\
		$\Rightarrow$ Die partielle Ableitung von u und v in a existieren und die 
		Jacobi-Matrix hat die Form:
		\[
		\begin{bmatrix}
		\frac{\partial u}{\p x}(a) & \frac{\p v}{\p x}(a)\\
		\frac{\partial u}{\p y}(a) & \frac{\p v}{\p y}(a)\\
		\end{bmatrix}	 
		\]
		\item
		Anschaulich
		%%%%%%%%%%%%%Lukas Bild1%%%%%%%%%%%%%%%%%%%%
		\begin{eqnarray*}
			f\prime(a) =&\lim_{h \to 0} \frac{f(a+h)-f(a)}{h}\\
			=& \lim_{h\to 0} \frac{f(a+ih)-f(a)}{ih}
		\end{eqnarray*}
		Wobei h immer reell sei,
		$\Rightarrow f\prime(a) = \p_1 u + i\p_1 v = \frac{1}{i}(\p_2 u 
		+ i \p_2 v)$
	\end{itemize}
\end{bemerkung}

\begin{satz}[Cauchy, Riemann]
	Sei $f:D\to \C, D\subset \C$, offen, $ a\in D$. Dann sind äquivalent:
	\begin{itemize}
		\item[1)]
		f ist in a komplex differenzierbar
		\item[2)]
		f ist in a total differenzierbar und für $u = Re(f), v = Im(f)$ gelten die 
		\textbf{Cauchy-Riemannschen-Differential Gleichungen}:
		\begin{eqnarray*}
			\frac{\p u}{\p x}(a) &=& \frac{\p v}{\p y}(a) \\
			\frac{\p u}{\p y}(a) &=& -\frac{\p v}{\p x}(a)
		\end{eqnarray*}
		Es gilt dann:
		\[
		f\prime(a) = \frac{\p u}{\p x}(a) + i \frac{\p v}{\p x}(a)
		= \frac{\p v}{\p y}(a) -i \frac{\p u}{\p y}(a)
		\]
		\item[3)]
		f ist in a total differenzierbar und die Tangentialabbildung ist $\C$-linear
	\end{itemize} 
\end{satz}

\begin{bew}
	1 $\Leftrightarrow$ 3: Bemerkung \ref{bemer} \\
	2 $\Leftrightarrow$ 3: Bemerkung \ref{bemer}
\end{bew}

\begin{satz}\label{satz}
	Seien $u,v:\R^2 \to \R$, stetig differenzierbar. Dann ist $f:=u+iv: D \to \C$ total 
	differenzierbar.\\
	Gilt ausserdem: $ \p_x u= \p_y v$, $\p_x v = -\p_y u$ überall in D, so ist 
	f in jedem Punkt von D komplex differenzierbar.
\end{satz}
\begin{bemerkung}
	Erste Aussage folgt aus Analysis II. Die zweite Aussage folgt aus dem Satz.
\end{bemerkung}
\begin{beispiel}
	\begin{itemize}
		\item[1)] Wir wissen $f(z) = z^2$ ist komplex differenzierbar
		$f(z) = z^2 = \underbrace{x^2 - y^2}_{u(x,y)}  + \underbrace{i 2xy}_{v(x,y)}$\\
		\[\p_x u = 2x = \p_yv = 2x\]
		\[\p_y u = -2y = -\p_xv = -2y\]
		\item[2)] $f(z)=x^3y^2+ix^2y^2$ ist überall total differenzierbar\\
		In $a = (\a,\b)$ gilt:
		\[ \p_xu(a) = 3\a^2\b^2 = \p_yv(a)=3\a^2\b^2 \]
		\[ \p_yu(a) = 2\a^3\b = -\p_xv(a)= -2\a\b^3 \]
		\[\Rightarrow \a\b(\a^2+\b^2)=0\]
		\[\Rightarrow \a=0 \text{ oder } \b = 0\]
		f ist nur auf den Koordinatenachsen differenzierbar
		\item[3)]
		$f(z) = \overline{z} = x +iy$
		\[\p_xu= 1 \ne \p_yv = -1\]
		f ist stetig in $\C$ aber nirgends komplex differenzierbar
		\item[4)]
		$\exp,\sin\cos$ sind in ganz $\C$ komplex differenzierbar: 
		\[\exp(z) = e^x\cos(y)+i e^x \sin(y)\]
		\[\p_xu = e^x\cos(y)= \p_y v\]
		\[\p_yu = -e^x\sin(y)= -\p_x v\]
		\[\exp\prime(z) =e^x \cos(y)+ie^x\sin(y)= \exp(z)\]
		
	\end{itemize}
\end{beispiel}

\textbf{Proposition:}\\
Alternative Formulierung von Satz \ref{satz}:\\
Sei $ f:D\to \C, D \subset \C$ offen, f stetig und total diffbar.\\
Wir definieren:
\[
\frac{\p f}{\p z}:= \frac{1}{2} (\frac{\p f}{\p x}-i\frac{\p f}{\p y}) 
\]
\[
\frac{\p f}{\p \overline{z}}:= \frac{1}{2} (\frac{\p f}{\p x}+i\frac{\p f}{\p y})
\]

$\frac{\p f}{\p z},\frac{\p \overline{f}}{\p \overline{z}}$ sind formale Ableitungen\\
Dann gilt: f ist komplex differenzierbar $\Leftrightarrow \left\{
\begin{array}{ll}
\frac{\p f}{\p \overline{z}}(a) = 0\\
\frac{\p f}{\p z}(a) = f(a)
\end{array}\right. \forall a \in D $
\begin{bew}
	nach Satz \ref{satz}:\\
	f ist komplex differenzierbar $\Leftrightarrow f = u+iv, \p_xu = \p_yv,\p_xv=-\p_yu$\\
	\[f = \p_xu +i\p_xv=\frac{1}{i}(\p_yu+i\p_yv) \Leftrightarrow f\prime=\p_xf=\frac{1}{i}\p_yf\]
	\[\Rightarrow \frac{\p f}{\p \overline{z}}=\frac{1}{2} (\p_xf+i\p_yf)= \frac{1}{2} (f\prime - f\prime)=0\]
	\[\frac{\p f}{\p z}=\frac{1}{2} (\p_xf-i\p_yf)= \frac{1}{2} (f\prime + f\prime)=f\prime\]
\end{bew}

\begin{definition}
	Eine Funktion $f:D\to \C$, offen, welche in jedem Punkt von D komplex differenzierbar ist, heißt auch \textbf{holomorph in D} (oder analytisch oder regulär).\\
	f ist holomorph in $ a\in D$, wenn es eine offene Umgebung $ U \subset D$ von a gibt , in der f holomorph ist.\\
	f heißt \textbf{antiholomorph in D}, falls $\overline{f}$ holomorph in D.\\
	\[
	\mathcal{O}(D):= \{f: D\subset \C|\text{ f holomorph in D}\}
	\]
\end{definition}

\begin{bemerkung}
	$\mathcal{O}(D) \subset$ C(D) ist $\C$-Unteralgebra.\\
	$e \in \mathcal{O}(D)$ ist Einheit $\Leftrightarrow$ e hat nirgends Nullstellen$ f \in \mathcal{O}(D) \Leftrightarrow \p_z f = 0$
\end{bemerkung}