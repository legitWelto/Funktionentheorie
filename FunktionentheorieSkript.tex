\documentclass[11pt]{article}
\usepackage{graphicx,color,amssymb,amsmath,amsthm}
\usepackage{url}
\usepackage{fullpage}
\usepackage[ngerman]{babel}
\usepackage[toc,page]{appendix}
\usepackage[utf8]{inputenc}
\usepackage{dsfont}
\parindent0mm

% sets of numbers
\def\R{\mathbb{R}}
\def\C{\mathbb{C}}
\def\N{\mathbb{N}}
\def\Q{\mathbb{Q}}
\def\Z{\mathbb{Z}}
% caligraphic letters
\def\cA{\mathcal{A}}
\def\cC{\mathcal{C}}
\def\cI{\mathcal{I}}
\def\cJ{\mathcal{J}}
\def\cK{\mathcal{K}}
\def\cL{\mathcal{L}}
\def\cN{\mathcal{N}}
\def\cR{\mathcal{R}}
\def\cS{\mathcal{S}}
\def\cT{\mathcal{T}}
% greek letters
\def\a{\alpha}
\def\b{\beta}
\def\g{\gamma}
\def\d{\delta}
\def\k{\kappa}
\def\l{\lambda}
\def\s{\sigma}
% partial derivatives
\def\p{\partial}
% other greek letters
\def\veps{\varepsilon}
\def\vrho{\varrho}
\def\vphi{\varphi}
% weak convergence
\def\wto{\rightharpoonup}
% Matlab
\def\matlab{{\sc Matlab}}
% domains and boundaries
\def\O{\Omega}
\def\G{\Gamma}
% differential operators
\DeclareMathOperator{\diver}{div}
% environements
\newtheorem{definition}{Definition}
\newtheorem{lemma}[definition]{Lemma}
\newtheorem{satz}[definition]{Satz}
\newtheorem{theorem}[definition]{Theorem}
\newtheorem{korollar}[definition]{Korollar}
\newtheorem{bemerkung}[definition]{Bemerkung}
\newtheorem{beispiel}[definition]{Beispiel}
\newtheorem{algorithmus}[definition]{Algorithmus}

\begin{document}
\title{Funktionentheorie}
\maketitle

\tableofcontents
\newpage

\section{Einführung}
Funktionentheorie ist Analysis, d.h. Differential- und Integralrechnung von Funktionen, die von einer komplexen Variablen abhängen, insbesondere von Funktionen, welche komplex differenzierbar sind.\\
Es gibt drei unterschiedliche, aber gleichwertige, sich ergänzende Sichtweisen.

Sei
$f: D \to \mathds{C},\quad  D \subset \mathds{C} \text{ offen}$
eine komplexe differenzierbare Fkt.\\
\begin{itemize}
	\item[1)] Cauchy (1814-1825) \\
	Jedes solche f besitzt eine natürliche Integraldarstellung
	\item[2)] Riemann (1851) \\
	Jede solche f kann als Abbildung zwischen Bereichen in der komplexen Zahlenebene $\mathds{C}$ aufgefasst werden, welche \textbf{konform} (=winkeltreu) ist.
	\item[3)] Weierstrass (1841) \\
	Jede solche f besitzt lokal eine konvergente \textbf{Potenzreihendarstellung}
\end{itemize}


\section{Anwendung in der (analytischen) Zahlentheorie}

Gauss (1792)\\
Sei $\pi(x) = |\{p \in \mathds{N}|\quad \text{p prim, } p \leq x\}|$ :\\
Suche Fkt. $f(x)$, sodass:
\[
\lim_{x \to \infty} \frac{\pi(x)-f(x)}{f(x)} = 0
\]
Vermutung:
\[
f(x) = \frac{log(x)}{x}
\]
Gauss wusste bereits, dass:
\[
\lim_{x \to \infty} \frac{\pi(x)}{x} = 0
\]
1896 (de la Vallee-Poussin, Hadamard)\\
Primzahlsatz:
\[
f(x) = \frac{\log(x)}{x}
\]
Beweis mit Hilfe der Funktionentheorie.\\
Riemann (1859)\\
Betrachte die Reihe:
\[
\zeta (s) = \sum^{\infty}_{n=1} \frac{1}{n^s}, s \in \mathds{C}
\]
Riemannsche Zeta-Funktion. Nullstellen von $\zeta (s)$ sind $\underbrace{s=-2,-4,-6,...}_{\text{triviale Nullstellen}}$\\
Vermutung(Riemann):\\
Alle anderen Nullstellen liegen auf $\{s\in \mathds{C}| Re(s) = \frac{1}{2}\}$\\
Millenium Prize Problem (Clay Math Institut, 2000).\\
Primzahlsatz $\Leftrightarrow $ Auf $\{s\in \mathds{C}| Re(s) = 1\}$ liegen keine Nullstellen.
\section{Differenzialrechnung im Komplexen}
\subsection{Komplexe Zahlen}
\begin{satz}
	Es gibt einen komplexen Körper $\mathds{C}$ mit den folgenden Eigenschaften:
	\begin{itemize}
		\item[1)]
		$\mathds{R}$ ist ein Unterkörper von $\mathds{C}$, d.h. $\mathds{R} \subset \mathds{C}$ und Addition, Multiplikation auf $\mathds{R}$ enstehen durch Einschränkung der Addition, Multiplikation auf $\mathds{C}$
		\item[2)]
		Die Geichung $ x^2 + 1 = 0$ besitzt genau zwei Lösungen
		\item[3)]
		Sei i eine der beiden Lösungen. dann ist die Abbildung:
		\begin{eqnarray*}
			\mathds{R} \times \mathds{R} \to  \mathds{C} \\
			(x,y) \mapsto x+ iy
		\end{eqnarray*}
		
		bijektiv.\\
	\end{itemize}
\end{satz}

\begin{bew}
Wir versehen die Menge $\mathds{C} := \mathds{R} \times \mathds{R}$ mit der Addition $(x,y) + (u,v) := (x+u,y+v)$ und mit der \\
Multiplikation $(x,y)* (u,v) = (xu-yv, xv+yu) \qquad \forall (x,y),(u,v) \in \mathds{R} \times \mathds{R}$\\
\end{bew}
\textbf{zz:}
\begin{itemize}
	\item[1)] Assoziativität
	\item[2)] Kommutativität 
	\item[3)] Distributivität 
	\item[4)] neutrales Element\\
	0 := (0,0), 1:= (1,0)
	\item[5)] inverses Element\\
	-z=(-x,-y), ...
\end{itemize}


$\mathds{R}$ ist Unterkörper von $\mathds{C}$. Es gilt:\\
\begin{eqnarray*}
	(a,0)(x,y) &=& (ax,ay)\\
	\text{d.h. }(a,0)(b,0)&=&(ab,0)\\
	(a,0)+(b,0) &=& (a+b,0)
\end{eqnarray*}

$\Rightarrow \mathds{C}_{\mathds{R}}:= \{(a,0) \in \mathds{C}\}$ ist ein Unterkörper versehen mit Addition und Multiplikation durch Einschränkung, isomorph zu $\mathds{R}$.\\
Wir identifizieren (a,0) $\in \mathds{C}$ mit $a \in \mathds{R}$.\\

\begin{definition}
	$\mathds{C}$ heißt \textbf{Körper der komplexen Zahlen}.\\
	$i := (0,1)$ heißt \textbf{imaginäre Einheit}.
\end{definition}

In der Darstellung $z=x+iy \in \mathds{C}$, heißt x der \textbf{Realteil von z} und y der \textbf{Imaginärteil von z}.\\
Wir schreiben:
\begin{eqnarray*}
x= Re(z)\\
y = Im(z)
\end{eqnarray*}
Ist $Re(z) = 0$, dann heißt z \textbf{rein imaginär}.\\
Wir definieren: $\overline{z}= x-iy$ und nennen $\overline{z}$ die zu \textbf{z konjugiert komplexe Zahl}.\\
\begin{itemize}
	\item $\overline{\overline{z}} = z$
	\item $\overline{z \pm w} = \overline{z} \pm \overline{w}$
	\item $Re(z) = \frac{1}{2}(z + \overline{z}), \qquad Im(z)= \frac{1}{2}i(z-\overline{z})$
	\item $z=\overline{z} \Leftrightarrow z \in \mathds{R}\qquad z=-\overline{z} \Leftrightarrow z \in i\mathds{R}$
\end{itemize}

\begin{bemerkung}
\begin{eqnarray*}
	\text{für } z,w \in \mathds{C} \text{ und } n,m \in \mathds{N} \text{ gilt:}\\
	z^n &:=& \underbrace{z*z *\cdots z}_{n} \\
	z^0 &:=& 1\\
	z^{-n} &:=& (z^{-1})^{n}, n>0, z \neq 0\\
	\Rightarrow \qquad z^m z^n&=&z^{n+m}\\
	(z^m)^n &=& z^{nm}\\
	z^n w^n &=& (zw)^n\\
	(z+w)^n &=& \sum^n_{k=0} z^k w^{n-k}\\
\end{eqnarray*}

...\\
\end{bemerkung}

Die komplexe Konjugation ist also ein involutorischer Körperautomorphismus mit Fixkörper $\mathds{R}$.\\
\begin{definition}
	Da $z\overline{z} = x^2+y^2$ reell und nicht negativ ist, definieren wir den Betrag von z durch: $|z|:= \sqrt{z\overline{z}}$
\end{definition}

\textbf{Lemma: }\\
Es gilt:
\begin{itemize}
	\item
	$|z| \geq 0, |z| = 0 \Leftrightarrow z = 0$
	\item
	$|zw| = |z||w|$
	\item
	$|Re(z) \leq |z|, |Im(z)| \leq |z|$
	\item
	 $|w+z| \leq |w|+|z|$
	 \item
	 $||z|-|w|| \leq |z-w|$
	 \item
	 $z^{-1} = \frac{\overline{z}}{|z|^2}$
\end{itemize}
\textbf{Beweis: }\\
1) bis 3) klar, 4), 5): Übungsaufgabe 6.\\
6) $z^{-1} =  \frac{x}{x^2 +y^2}- \frac{iy}{x^2+y^2} = \frac{\overline{z}}{|z|^2}$\\

\textbf{Bemerkung:}\\
Geometrische Veranschaulichung: \\
\includegraphics[scale=0.1, angle=180]{pics/Polar.jpg} \\

\textbf{Proposition: }\\
\begin{eqnarray*}
\mathds{R}^x_+ &=& \{z \in \mathds{C}| x > 0\} \\
\mathds{C}^x &=& \{ z \in \mathds{C} | z \neq 0\} =  \mathds{C} \setminus\{0\} 
\end{eqnarray*}

\begin{itemize}
	\item
	Die Polarkoordinatendarstellungsabbildung:
	\begin{eqnarray*}
	\mathds{R}^x_+ \times \mathds{C} \\
	(r,\varphi) \mapsto z = r(cos(\varphi),i sin(\varphi))
	\end{eqnarray*}
	ist surjektiv.
	\item
	Aus $z = z', d.h.\quad r(cos(\varphi, i sin(vaphi)) = r'(cos(\varphi', i sin(\varphi'))$\\
	...\\$r(cos(\varphi, i sin(\varphi))$
\end{itemize}
\textbf{Beweis:}
Aus reellen Analysis folgt: \\
jedes $z = (x,y) \neq (0,0)$ kann eindeutig geschrieben werden als $z = (r cos(\varphi), rsin(\varphi))$. \\
Daher ist r eindeutig bestimmt durch $r = \sqrt{x^2 + y^2}$ \\
$\varphi$ ist jedoch nur eindeutig bis auf ganzzahlig Vielfache von $2\pi$

\begin{definition}
	Sei $z \in \mathds{C}$. Jedes $\varphi \in  \mathds{R}$ mit $z =r(cos(\varphi, i sin(\varphi))  $ heißt ein \textbf{Argument von z}. Wir schreiben $\varphi = arg(z)$.
	Falls $\varphi \in (\-\pi,\pi]$ , heißt $\varphi$ \textbf{Hauptwert des Arguments von z}. Wir schreiben $\varphi = Arg(z)$.
\end{definition}

\subsection{Einführung}

\begin{beispiel}
$z = \frac{1}{1+i} = \frac{1}{2} - \frac{i}{2}$\\
$|z| = \frac{1}{\sqrt{2}}, z = \frac{1}{\sqrt{2}}(\cos(-\pi/4)+i\sin(-\pi/4))$
\end{beispiel}

\begin{lemma}
\[
(\cos\phi + i\sin\phi)(\cos\phi\prime + i\sin\phi\prime))
\]
\[ \text{m.a.W}. arg(z*z\prime) = arg(z)+ arg(z\prime)
\]
\end{lemma}

\begin{bew}
Additionstheorem der Winkelfkt.
\end{bew}
\begin{bemerkung}
%%%%%%%Lukas Bild 1%%%%%%
\end{bemerkung}

\begin{definition}
$z \in \C $ heißt \textbf{m-te Einheitswurzel}, falls gilt $z^m = , m \in \N$
\end{definition}
\begin{satz}
Zu jedem $m \in \N$ gibt es genau m verschieden m-te Einheitswurzeln:
\[
\zeta^m_k := \cos(2\pi\frac{k}{m}+i\sin(2\pi\frac{k}{m}), k=0,\dots,m-1
\]
Beweis Übungsaufgabe 3.
\end{satz}

\subsection{Konvergente Folgen und Reihen}
\begin{definition}
Eine Folge $(z)_{nin \N}, z_n \in \C$ heißt \textbf{Nullfolge}, falls $\forall \epsilon > 0 \exists N \in \N$ mit $|z_n| < \epsilon \forall n>N.$ \\
Eine Folge $(z_n)_{n\in \N}, z_n \in \C$ \textbf{konvergiert gegen z} $\in \C$
falls die Folge $(z_n-z)_{n\in \N}$ ein Nullfolge ist. Der Grenzwert ist eindeutig bestimmt: $z = \lim_{n \to \infty}z_n$
\end{definition}

\begin{bemerkung}
%%%%%%Lukas Bermerkung 1%%%%%%
\end{bemerkung}

\begin{definition}
Sei $(z_n)_{n\in\N}, z_n \in \C$, eine Folge.\\
die Folge der Partialsummen $(S_n)_{n\geq 0}$ mit $S_n := z_0+\dots+z_n$ heißt die der \textbf{Folge $(z_n)$ zugeordnete Reihe}.
Falls $(S_n)$ konvergert, so heißt $ S:=\lim_{n\to \infty} S_n$ Wert der Reige und wir schreiben $\sum^\infty_{n=0}z_n := S.$\\
Eine Reihe $\sum^\infty_{n=0} z_n$ heißt \textbf{absolut konvergent} falls die Reihe 
$\sum |z_n|$ konvergiert.
\end{definition}

\begin{beispiel}
%%%%%%%Lukas Beipiel 1
\end{beispiel}

\begin{satz}
Eine absolut konvergente Reihe konvergiert.
\end{satz}
\begin{bew}
Analysis 1 und Bemerkung 2.2.1
\end{bew}

\begin{definition}
Da die Reihe
\[
\sum \frac{z^n}{n!}, \sum \frac{(-1)^n z^{2n}}{(2n)!}, \sum \frac{(-1)^nz^{2n+1}}{(2n+1)!}
\]
absolut konvergent für alle $z \in \C$ definieren wir: (exp,cos,sin...)
\end{definition}

\begin{lemma}[Multiplikatinossatz von Cauchy]
Summe immer von 0 bis $\infty$ \\
Seien $\sum z_n, \sum w_m$ absolut konvergente Reihen dann gilt:
\[
(\sum z_n)(\sum w_m) = \sum \sum z_k w_n-k
\]
und die rechte Seite ist ebenfalls absolut konvergent.
\end{lemma}
\begin{bew}
Kopie des Bew. aus Analysis 1
\end{bew}

\begin{satz}
\[
\exp(z) \exp(w) = exp(z+w) \forall z,w \in \C
\]
\end{satz}
\begin{bew}
\[
\sum \frac{z^n}{n!} \sum \frac{w^m}{m!} = \sum \sum \frac{z^kw-k}{n!(n-k)!} = \sum \frac{(z+w)^n}{n!}
\]
\end{bew}

\begin{korollar}
\begin{eqnarray*}
\exp(z) &\neq& 0\\
\exp(z)^n &=& exp(nz)\\
\exp(iz) &=& \cos(z) + i \sin(z)\\
\cos z &=& 1/2 (\exp(iz)+\exp(-iz))\\
\sin z &=& 1/(2i) (\exp(iz)-\exp(-iz))\\
\text{falls z = x+iy} \exp z &=& e^x ( \cos y + i \sin y) \\
|\exp z | &=& e^x\\
\cos(z+w) &=& \cos(z)\cos(w) - \sin(z) \sin(w)\\
\sin(z+w) &=& \cos z \sin w + \sin z \cos w
\end{eqnarray*}
\end{korollar}

\begin{bemerkung}
$\exp(z)$ ist nicht injektiv, da $\exp(2\pi i k ) = 1 \forall k \in \Z$.\\
Genauer $ \exp(z) = exp(w) \Leftrightarrow z-w \in 2\pi i \Z$ \\
also $ker \exp = \{z \in \C| \exp z = 1\} = 2\pi i \Z$
$\exp( z+w) = \exp z \forall z \in \C \Leftrightarrow w \in ker exp$.\\
Die Expfkt. \textbf{ist periodisch} mit Perioden $2 \pi i k, k \in \Z$.\\
\begin{eqnarray*}
\sin z = 0 &\Leftrightarrow & z = k\pi, k \in \Z \\
\cos z = 0 &\Leftrightarrow & z = (k+ 1/2)\pi, k \in \Z
\end{eqnarray*}
\end{bemerkung}

Sei
\[ 
S = \{ w \in \C| -\pi < Im(w) \leq \pi\}
\]
$\exp|_S$ ist injektiv, wegen Periodizität
\begin{eqnarray*}
\{\exp(z)| z \in S\} = \{ \exp z | z \in \C\}\\
\Rightarrow \{ \exp z | z \in \C \} = \C^{\times}\\
\exp|_S: S \to \C^{\times}\\
z \mapsto exp z \text{ ist bijektiv}
\end{eqnarray*}

\begin{definition}
%%%%%%%%%Lukas Definition 1%%%%%%%%%%
\end{definition}
\begin{satz}
1)\\
Es exitiert eine Fkt., der sogenannte \textbf{Hauptzweig des Log}, $ Log: \C^{\times} \to \C$ , welche durch $ exp(Log z) = z und -\pi< Im(Log z) \leq \pi$ eindeutig bestimmt ist.\\
2)\\
$\exp w = z \Rightarrow w = Log z + 2 \pi i , k \in \Z$\\
3)\\
$Log z = log z$ falls $ z \in \R^{\times}_+$\\
4)\\
$Log z = log(|z|) + i Arg(z)$\\
\end{satz}
\begin{bew}
1) folgt aus Bem. 2.10, ebenso 2) aus 1) und Bem. 2.10, 3) klar, 4) folgt aus 1), Kor 2.9 und 1.8
\end{bew}
\subsection{Komplexe Ableitung und Cauchy-Riemannsche Differentialgleichung}
\textbf{Erinnerung:}\\
Eine Funktion $f:D \to \C, D \subset \C$ ist stetig, wenn sie stetig ist aufgefasst als Abbildung einer Teilmenge $D\subset \R^2$ in den $\R^2$.\\
Eine Teilmenge $D\subset \R^2$ ist offen, falls $\forall q \in D \quad \exists \veps > 0: \cup _\veps(a) = \{z \in \R^2| |z-a| < \veps\} \subset D$.\\
$a \in \C$ ist Häufungspunkt von D, falls $ \forall \veps > 0 \quad \exists z \in D$ mit $|z-a| < \veps$ .\\
Sei $f: D \to \C, L \in \C,$ dann bedeutet die Schreibweise $\begin{displaystyle}
\lim_{z \to a} f(z)=l 
\end{displaystyle}$, dass a ein Häufungspunkt von D ist und $\tilde{f} : D \cup \{a\} \to \C, z\mapsto \left\{
\begin{array}{ll}
	f(z)&z\neq a\\
	l & z=a 
\end{array}\right. $ ist in a stetig


\begin{definition}
Eine Funktion $f: D \to \C, \quad D \subset \C$ offen, heißt \textbf{komplex differenzierbar im Punkt $a\in D$}, falls:
\[
\lim_{z \to a} \frac{f(z) -f(a)}{z-a}
\]
existiert. Falls der Grenzwert existiert schreiben wir für den Grenzwert:
\[
f\prime(a) = \frac{df}{dz}(a)
\]
Wenn f in jedem Punkt komplex differenzierbar ist, dann ist $f\prime$ wieder eine Funktion auf D.
\end{definition}

\begin{bemerkung}
Alternative Formulierung der komplexen Differenzierbarkeit:
Die folgenden Aussagen sind äquivalent:
\begin{itemize}
	\item[1)] 
	f ist in $a\in D$ komplex differenzierbar und hat den Grenzwert l
	\item[2)]
	$\exists f_1$, stetig in a, mit $f(z) = f_1(z)(z-a), f(a) = l$
	\item[3)]
	$\exists g: D \to \C$, stetig in a, mit $f(z) = f(a) + (z-a)l+ g(z), g(a)=0$
	\item[4)]
	$\exists r: D\to \C$, stetig in a, mit $f(z) = f(a) +(z-a)l + r(z), \lim_{z \to a}\frac{r(z)}{z-a}= 0$\\
	Insbesondere folgt: f ist komplex differenzierbar in a $\Rightarrow$ f stetig in a.
\end{itemize}
\end{bemerkung}

\begin{satz}
Seien $f,g: D \to \C, D\subset \C$, offen, komplex differenzierbar auf D, Dann sind, f+g, $\lambda$f ($\lambda \in \C$), $f \cdot g, \frac{f}{g}$ ($g(a) \neq 0$) ebenfalls komplex differenzierbar. Es gelten die üblichen Ableitungsregeln(Ketten-,Produkt-,Quotientenregel).
\end{satz}
\begin{beispiel}
	\leavevmode
	\begin{itemize}
		\item[1)]
		$f: D \to \C, z\mapsto z^n, n \in \Z  $\\
		ist komplex differenzierbar mit $f\prime(z) = n z^{n-1}$ 
		$D = \left\{
		\begin{array}{ll}
		\C &n > 0\\
		\C^{\times} &n \leq 0
		\end{array}\right. $ \\
		Jedes Polynom $ P(z) = \sum_{k=0}^{n} a_k z^k ,a_k \in \C , a_n \ne 0$ ist komplex differenzierbar und es gilt: $P\prime(z) = \sum^{n - 1}_{k=0} k a_k z^{k-1}$ \\
		Sei $Q(z)$ ein weiteres Polynom\\
		$N(Q) = \{z \in \C| Q(z)=0\}$\\
		Dann ist $f :\C \setminus n(Q) \to \C$ komplex differenzierbar\\
		$z \mapsto \frac{P(z)}{Q(z)}$
		\item[2)]
		Sei $(a_n)_{n \in \N}, a_n \in \N$ eine Folge.\\
		Annahme: Die Potenzreihe $f(z) = \sum^\infty_{n=0} a_n(z-a)^n$
		konvergieren in \\
		$B_{R}(a)=\{z \in \C| |z-a|<R\} R>0\}$ \\
		Wir werden später sehen, f ist in jedem Punkt von $B_R(a)$ 
		komplex differenzierbar\\
	\end{itemize}
\end{beispiel}


\begin{bemerkung}\label{bemer}
	\leavevmode
	\begin{itemize}
		\item
		Erinnerung aus Analysis II :\\
		Eine Abbildung von D nach $\R^a, D \subset \R^p$, offen, heißt total
		differenzierbar in $a \in D$, falls es eine $\R$-lineare Abbildung 
		$T: \R^p \to \R^q$ gibt, mit:\\
		$f(x)-f(a) = T(x-a) +r(x), \lim_{x \to a} \frac{r(x)}{|x-a|}= 0$\\
		T ist eindeutig bestimmt und heißt Jacobi-Abbildung von f in a 
		oder Tangentialabbildung von f in a oder totales Differential. Wir 
		schreiben: \\
		$Tf(a) := T$ \\
		$ f: D \rightarrow \C , D \subset \C $ offen ist komplex differenzierbar in $a \in D \Leftrightarrow$ f ist in a total differenzierbar ($\C \tilde{=} \R^2$ ) und die Tangentialabbildung ist von der Form : $Tf(a)z = lz$ mit $l = f\prime(a) \in \C$ 
		\item
		Erinnerung aus der Linearen Algebra:\\
		Für eine $\R$-lineare Abb. $T: \C \to \C$ sind folgende Aussagen äquivalent:
		\begin{itemize}
			\item[1)]
			$\exists l \in \C \text{ mit } Tz = lz$
			\item[2)]
			T ist $\C$-linear
			\item[3)]
			$T(i) = iT(1)$
			\item[4)]
			Die Matrix von T bezüglich der Basis $1(=(1,0))$ und die Form \\
			i(=(0,1)) hat die Form:
			\[
			\begin{bmatrix}
			\a & -\b\\
			\b & \a
			\end{bmatrix}
			\]
		\end{itemize}
		\begin{bew}
			$1 \Leftrightarrow 4$ Wir können die Matrix von t schreiben als:
			\[ \begin{bmatrix}
			a & b \\
			c & d
			\end{bmatrix}
			\rightarrow \begin{bmatrix}
			u \\
			v
			\end{bmatrix} = 
			\begin{bmatrix}
			a & b \\
			c & d
			\end{bmatrix}
			\begin{bmatrix}
			x \\
			y
			\end{bmatrix}\]
			\begin{eqnarray*}
				\text{aus } Tz = lz, l= \a + i\b, z = x + iy = (x,y) \\
				\Rightarrow T(x,y) =& (\a x - \b y, \a y + \b x)\\
				=& \begin{bmatrix}
					a & b \\
					c & d
				\end{bmatrix}
				\begin{bmatrix}
					x \\
					y
				\end{bmatrix}
			\end{eqnarray*}
			
		\end{bew}
		\item
		Erinnerung aus Analysis II:\\
		Die Matrix der Tangentialabbildung ist die Jacobi-Matrix der 
		partiellen Ableitungen.\\
		$f:D \to \C, D\subset \C$ offen, total differenzierbar in $a \in D$.\\
		$f(x,y) = u(x,y) +iv(x,y), z = x+iy=(x,y)$\\
		$\Rightarrow$ Die partielle Ableitung von u und v in a existieren und die 
		Jacobi-Matrix hat die Form:
		\[
		\begin{bmatrix}
		\frac{\partial u}{\p x}(a) & \frac{\p v}{\p x}(a)\\
		\frac{\partial u}{\p y}(a) & \frac{\p v}{\p y}(a)\\
		\end{bmatrix}	 
		\]
		\item
		Anschaulich
		%%%%%%%%%%%%%Lukas Bild1%%%%%%%%%%%%%%%%%%%%
		\begin{eqnarray*}
			f\prime(a) =&\lim_{h \to 0} \frac{f(a+h)-f(a)}{h}\\
			=& \lim_{h\to 0} \frac{f(a+ih)-f(a)}{ih}
		\end{eqnarray*}
		Wobei h immer reell sei,
		$\Rightarrow f\prime(a) = \p_1 u + i\p_1 v = \frac{1}{i}(\p_2 u 
		+ i \p_2 v)$
	\end{itemize}
\end{bemerkung}

\begin{satz}[Cauchy, Riemann]
	Sei $f:D\to \C, D\subset \C$, offen, $ a\in D$. Dann sind äquivalent:
	\begin{itemize}
		\item[1)]
		f ist in a komplex differenzierbar
		\item[2)]
		f ist in a total differenzierbar und für $u = Re(f), v = Im(f)$ gelten die 
		\textbf{Cauchy-Riemannschen-Differential Gleichungen}:
		\begin{eqnarray*}
			\frac{\p u}{\p x}(a) &=& \frac{\p v}{\p y}(a) \\
			\frac{\p u}{\p y}(a) &=& -\frac{\p v}{\p x}(a)
		\end{eqnarray*}
		Es gilt dann:
		\[
		f\prime(a) = \frac{\p u}{\p x}(a) + i \frac{\p v}{\p x}(a)
		= \frac{\p v}{\p y}(a) -i \frac{\p u}{\p y}(a)
		\]
		\item[3)]
		f ist in a total differenzierbar und die Tangentialabbildung ist $\C$-linear
	\end{itemize} 
\end{satz}

\begin{bew}
	1 $\Leftrightarrow$ 3: Bemerkung \ref{bemer} \\
	2 $\Leftrightarrow$ 3: Bemerkung \ref{bemer}
\end{bew}

\begin{satz}\label{satz}
	Seien $u,v:\R^2 \to \R$, stetig differenzierbar. Dann ist $f:=u+iv: D \to \C$ total 
	differenzierbar.\\
	Gilt ausserdem: $ \p_x u= \p_y v$, $\p_x v = -\p_y u$ überall in D, so ist 
	f in jedem Punkt von D komplex differenzierbar.
\end{satz}
\begin{bemerkung}
	Erste Aussage folgt aus Analysis II. Die zweite Aussage folgt aus dem Satz.
\end{bemerkung}
\begin{beispiel}
	\begin{itemize}
		\item[1)] Wir wissen $f(z) = z^2$ ist komplex differenzierbar
		$f(z) = z^2 = \underbrace{x^2 - y^2}_{u(x,y)}  + \underbrace{i 2xy}_{v(x,y)}$\\
		\[\p_x u = 2x = \p_yv = 2x\]
		\[\p_y u = -2y = -\p_xv = -2y\]
		\item[2)] $f(z)=x^3y^2+ix^2y^2$ ist überall total differenzierbar\\
		In $a = (\a,\b)$ gilt:
		\[ \p_xu(a) = 3\a^2\b^2 = \p_yv(a)=3\a^2\b^2 \]
		\[ \p_yu(a) = 2\a^3\b = -\p_xv(a)= -2\a\b^3 \]
		\[\Rightarrow \a\b(\a^2+\b^2)=0\]
		\[\Rightarrow \a=0 \text{ oder } \b = 0\]
		f ist nur auf den Koordinatenachsen differenzierbar
		\item[3)]
		$f(z) = \overline{z} = x +iy$
		\[\p_xu= 1 \ne \p_yv = -1\]
		f ist stetig in $\C$ aber nirgends komplex differenzierbar
		\item[4)]
		$\exp,\sin\cos$ sind in ganz $\C$ komplex differenzierbar: 
		\[\exp(z) = e^x\cos(y)+i e^x \sin(y)\]
		\[\p_xu = e^x\cos(y)= \p_y v\]
		\[\p_yu = -e^x\sin(y)= -\p_x v\]
		\[\exp\prime(z) =e^x \cos(y)+ie^x\sin(y)= \exp(z)\]
		
	\end{itemize}
\end{beispiel}

\textbf{Proposition:}\\
Alternative Formulierung von Satz \ref{satz}:\\
Sei $ f:D\to \C, D \subset \C$ offen, f stetig und total diffbar.\\
Wir definieren:
\[
\frac{\p f}{\p z}:= \frac{1}{2} (\frac{\p f}{\p x}-i\frac{\p f}{\p y}) 
\]
\[
\frac{\p f}{\p \overline{z}}:= \frac{1}{2} (\frac{\p f}{\p x}+i\frac{\p f}{\p y})
\]

$\frac{\p f}{\p z},\frac{\p \overline{f}}{\p \overline{z}}$ sind formale Ableitungen\\
Dann gilt: f ist komplex differenzierbar $\Leftrightarrow \left\{
\begin{array}{ll}
\frac{\p f}{\p \overline{z}}(a) = 0\\
\frac{\p f}{\p z}(a) = f(a)
\end{array}\right. \forall a \in D $
\begin{bew}
	nach Satz \ref{satz}:\\
	f ist komplex differenzierbar $\Leftrightarrow f = u+iv, \p_xu = \p_yv,\p_xv=-\p_yu$\\
	\[f = \p_xu +i\p_xv=\frac{1}{i}(\p_yu+i\p_yv) \Leftrightarrow f\prime=\p_xf=\frac{1}{i}\p_yf\]
	\[\Rightarrow \frac{\p f}{\p \overline{z}}=\frac{1}{2} (\p_xf+i\p_yf)= \frac{1}{2} (f\prime - f\prime)=0\]
	\[\frac{\p f}{\p z}=\frac{1}{2} (\p_xf-i\p_yf)= \frac{1}{2} (f\prime + f\prime)=f\prime\]
\end{bew}

\begin{definition}
	Eine Funktion $f:D\to \C$, offen, welche in jedem Punkt von D komplex differenzierbar ist, heißt auch \textbf{holomorph in D} (oder analytisch oder regulär).\\
	f ist holomorph in $ a\in D$, wenn es eine offene Umgebung $ U \subset D$ von a gibt , in der f holomorph ist.\\
	f heißt \textbf{antiholomorph in D}, falls $\overline{f}$ holomorph in D.\\
	\[
	\mathcal{O}(D):= \{f: D\subset \C|\text{ f holomorph in D}\}
	\]
\end{definition}

\begin{bemerkung}
	$\mathcal{O}(D) \subset$ C(D) ist $\C$-Unteralgebra.\\
	$e \in \mathcal{O}(D)$ ist Einheit $\Leftrightarrow$ e hat nirgends Nullstellen$ f \in \mathcal{O}(D) \Leftrightarrow \p_z f = 0$
\end{bemerkung}

\textbf{Beispiele}
\begin{itemize}
	\item
	$f: D \to \C, z\mapsto z^n, n \in \Z  $\\
	ist komplex diffbar. mit $f\prime(z) = n z^{n-1}$\\
	D = \{$\C n > 0, \C^{\times} n \leq 0$\}
	Jedes Polynom ist komplex diffbar.
	Sei $Q(z)$ ein weiteres Polynom\\
	$N(Q) = \{z \in \C| Q(z)=0\}$\\
	Dann ist $f :\C \setminus n(Q) \to \C$ komplex diffbar\\
	$z \mapsto \frac{P(z)}{Q(z)}$
	\item
	Sei $(a_n)_{n \in \N}, a_n \in \N$ eine Folge.\\
	Annahme: Die Potenzreihe $f(z) = \sum^\infty_{n=0} a_n(z-a)^n$
	konvergieren in \\
	$B_{R}(a)=\{z \in \C| |z-a|<R\} R>0\}$ \\
	Wir werden später sehen f ist in jedem Punkt von $B_R(a)$ 
	komplex diffbar\\
\end{itemize}

\begin{bemerkung}
\leavevmode
\begin{itemize}
	\item
	Erinnerung aus Ana 2:\\
	Eine Abbildung von D nach $\R^a, D \subset \R^p$, offen, heißt total
	diffbar in $a \in D$, falls eine $\R$-lineare Abbildung \\
	$T: \R^p \to \R^q$ gibt, mit:\\
	$f(x)-f(a) = T(x-a) +r(x), \lim_{x \to a} \frac{r(x)}{|x-a|}= 0$\\
	T ist eindeutig bestimmt und heißt Jacobi-Abb.von f in a 
	oder Tangentialabb. von f in a oder totales Differential. Wir 
	schreiben: \\
	$Tf(a) := T$ \\
	%%%%%%%%%%%%%Lukas Bemerkung 1%%%%%%%%%%%%%%%%%%%%%%
	\item
	Erinnerung aus LA:\\
	Für eine $\R$-lineare Abb. $T: \C \to \C$ sind folgende Aussagen äqiv.:
	\begin{itemize}
		\item
		$\exists l \in \C \text{ mit } Tz = lz$
		\item
		T ist $\C$-linear
		\item
		$T(i) = iT(1)$
		\item
		Die Matrix von T bezüglich der Basis $1(=(1,0))$ und die Form \\
		i(=(0,1)) hat die Form:
		\[
		\begin{bmatrix}
		\a & -\b\\
		\b & \a
		\end{bmatrix}
		\]
		%%%%%%%%%%%%%%%Lukas Beweis 1%%%%%%%%%%%%%%%%%
	\end{itemize}
	\item
	Erinnerung aus Ana 2:\\
	Die Matrix der Tangentialab ist die Jacobi-Matrix der 
	partiellen Ableitungen.\\
	$f:D \to \C, D\subset \C$ offen, total diffbar in $a \in D$.\\
	$f(x,y) = u(x,y) +iv(x,y), z = x+iy=(x,y)$\\
	$\Rightarrow$ Die part. Abl. von u und v in a existieren und die 
	Jacobi-Matrix hat die form:
	\[
	\begin{bmatrix}
	\frac{\partial u}{\p x}(a) & \frac{\p v}{\p x}(a)\\
	\frac{\partial u}{\p y}(a) & \frac{\p v}{\p y}(a)\\
	\end{bmatrix}	 
	\]
	\item
	Anschaulich
	%%%%%%%%%%%%%Lukas Bild1%%%%%%%%%%%%%%%%%%%%
	\begin{eqnarray*}
	f\prime(a) =&\lim_{h \to 0} \frac{f(a+h)-f(a)}{h}\\
	=& \lim_{h\to 0} \frac{f(a+ih)-f(a)}{ih}
	\end{eqnarray*}
	Wobei h immer reell sei,
	$\Rightarrow f\prime(a) = \p_1 u + i\p_1 v = \frac{1}{i}(\p_2 u 
	+ i \p_2 v)$
\end{itemize}
\end{bemerkung}

\begin{satz}[Cauchy, Riemann]
Sei $f:D\to \C, D\subset \C$, offen, $ a\in D$. Dann sind äquiv.
\begin{itemize}
	\item
	f ist in a komplex diffbar
	\item
	f ist in a total diffbar und für $u = Re(f), v = Im(f)$ gelten die 
	\textbf{Cauchy-Riemannschen-DGLs}:
	\begin{eqnarray}
	\frac{\p u}{\p x}(a) = \frac{\p v}{\p y}(a) \\
	\frac{\p u}{\p y}(a) = -\frac{\p v}{\p x}(a)
	\end{eqnarray}
	Es gilt dann:
	\[
	f\prime(a) = \frac{\p u}{\p x}(a) + i \frac{\p v}{\p x}(a)
	= \frac{\p v}{\p y}(a) -i \frac{\p u}{\p y}(a)
	\]
	\item
	F ist in a total diffbar. und die Tangentialabb. ist $\C$-linear
\end{itemize} 
\end{satz}

\begin{bew}
 1 gdw 3: bem vorher
 2 gdw 3 bem vorher 
\end{bew}

\begin{satz}
Seien $u,v:\R^2 \to \R$, stetig diffbar. Dann ist $f:=u+iv: D \to \C$ total 
diffbar.\\
Gilt ausserdem: $ \p_x u= \p_y v$, $\p_x v = -\p_y u$ überall in D, so ist 
f in jedem Punkt von D komplex diffbar.
\end{satz}
\begin{bemerkung}
Erste Aussage folgt aus Ana 2. Die zweite Aussage folgt aus dem Satz.
\end{bemerkung}
%%%%%%%%%%%%%%%%%%%Lukas BSp. 1%%%%%%%%%%%%%%%%%%%%%%%%

\textbf{Proposition:}\\
Alternative formulierung von Satz -3 ??:\\
Sei $ f:D\to \C, D \subset \C$ offen, f stetig und total diffbar.\\
Wir definieren:
\[
\frac{\p f}{\p z}:= \frac{1}{2} (\frac{\p f}{\p x}-i\frac{\p f}{\p y}) 
\]
\[
\frac{\p f}{\p \overline{z}}:= \frac{1}{2} (\frac{\p f}{\p x}+i\frac{\p f}{\p y})
\]

%%%%%%%%%%%%%%%%%%%%%%Lukas Proposition 1%%%%%%%%%%%%%%%%%%%%

\begin{definition}
Eine Funktion $f:D\to \C$, offen, welche in jedem Punkt von D komplex diffbar ist, heißt auch \textbf{holomorph in D} ( oder analytisch oder regulär).\\
f ist holomorph in $ a\in D$, wenn es eine offene Umgebung $ U \subset D$ von a gibt , in der f holomorph ist.\\
f heißt \textbf{antiholomorph in D}, falls $\overline{f}$ holomorph in D.\\
\[
O(D):= \{f: D\subset \C|\text{ f holomorph in D}\}
\]
\end{definition}

\begin{bemerkung}
O(D) in C(D) ist $\C$-Unteralgebra.\\
$e \in O(D)$ ist Einheit $\Leftrightarrow$ e hat nirgends NST$ f \in O(D) \Leftrightarrow \p_z f = 0$
\end{bemerkung}
\end{document}